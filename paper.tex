% Options for packages loaded elsewhere
\PassOptionsToPackage{unicode}{hyperref}
\PassOptionsToPackage{hyphens}{url}
\PassOptionsToPackage{dvipsnames,svgnames,x11names}{xcolor}
%
\documentclass[
  xelatex,
  ja=standard,
  12pt,
  a4paper]{bxjsarticle}

\usepackage{amsmath,amssymb}
\usepackage{iftex}
\ifPDFTeX
  \usepackage[T1]{fontenc}
  \usepackage[utf8]{inputenc}
  \usepackage{textcomp} % provide euro and other symbols
\else % if luatex or xetex
  \usepackage{unicode-math}
  \defaultfontfeatures{Scale=MatchLowercase}
  \defaultfontfeatures[\rmfamily]{Ligatures=TeX,Scale=1}
\fi
\usepackage{lmodern}
\ifPDFTeX\else  
    % xetex/luatex font selection
\fi
% Use upquote if available, for straight quotes in verbatim environments
\IfFileExists{upquote.sty}{\usepackage{upquote}}{}
\IfFileExists{microtype.sty}{% use microtype if available
  \usepackage[]{microtype}
  \UseMicrotypeSet[protrusion]{basicmath} % disable protrusion for tt fonts
}{}
\makeatletter
\@ifundefined{KOMAClassName}{% if non-KOMA class
  \IfFileExists{parskip.sty}{%
    \usepackage{parskip}
  }{% else
    \setlength{\parindent}{0pt}
    \setlength{\parskip}{6pt plus 2pt minus 1pt}}
}{% if KOMA class
  \KOMAoptions{parskip=half}}
\makeatother
\usepackage{xcolor}
\setlength{\emergencystretch}{3em} % prevent overfull lines
\setcounter{secnumdepth}{3}
% Make \paragraph and \subparagraph free-standing
\makeatletter
\ifx\paragraph\undefined\else
  \let\oldparagraph\paragraph
  \renewcommand{\paragraph}{
    \@ifstar
      \xxxParagraphStar
      \xxxParagraphNoStar
  }
  \newcommand{\xxxParagraphStar}[1]{\oldparagraph*{#1}\mbox{}}
  \newcommand{\xxxParagraphNoStar}[1]{\oldparagraph{#1}\mbox{}}
\fi
\ifx\subparagraph\undefined\else
  \let\oldsubparagraph\subparagraph
  \renewcommand{\subparagraph}{
    \@ifstar
      \xxxSubParagraphStar
      \xxxSubParagraphNoStar
  }
  \newcommand{\xxxSubParagraphStar}[1]{\oldsubparagraph*{#1}\mbox{}}
  \newcommand{\xxxSubParagraphNoStar}[1]{\oldsubparagraph{#1}\mbox{}}
\fi
\makeatother


\providecommand{\tightlist}{%
  \setlength{\itemsep}{0pt}\setlength{\parskip}{0pt}}\usepackage{longtable,booktabs,array}
\usepackage{calc} % for calculating minipage widths
% Correct order of tables after \paragraph or \subparagraph
\usepackage{etoolbox}
\makeatletter
\patchcmd\longtable{\par}{\if@noskipsec\mbox{}\fi\par}{}{}
\makeatother
% Allow footnotes in longtable head/foot
\IfFileExists{footnotehyper.sty}{\usepackage{footnotehyper}}{\usepackage{footnote}}
\makesavenoteenv{longtable}
\usepackage{graphicx}
\makeatletter
\newsavebox\pandoc@box
\newcommand*\pandocbounded[1]{% scales image to fit in text height/width
  \sbox\pandoc@box{#1}%
  \Gscale@div\@tempa{\textheight}{\dimexpr\ht\pandoc@box+\dp\pandoc@box\relax}%
  \Gscale@div\@tempb{\linewidth}{\wd\pandoc@box}%
  \ifdim\@tempb\p@<\@tempa\p@\let\@tempa\@tempb\fi% select the smaller of both
  \ifdim\@tempa\p@<\p@\scalebox{\@tempa}{\usebox\pandoc@box}%
  \else\usebox{\pandoc@box}%
  \fi%
}
% Set default figure placement to htbp
\def\fps@figure{htbp}
\makeatother

% NAIST修士論文用LaTeXヘッダー(Quarto用)
% senshuQmdスタイルを参考に、NAISTフォーマットに対応

% Quartoのデフォルト\maketitleを先にバックアップ(NAISTの\titlepageを使用するため)
\makeatletter
\@ifundefined{maketitle}{}{%
  \let\quarto@maketitle\maketitle
}
\makeatother

% NAISTのスタイルファイルを読み込む
% template/naist-jmthesis.styを使用(\input naist-mcommon.styを\input{Mtex/naist-mcommon.sty}に修正済み)
\makeatletter
\input{template/naist-jmthesis.sty}
\makeatother

% NAISTフォーマットのページレイアウトを設定(スタイルファイルの後に設定して上書き)
\usepackage{geometry}
\geometry{
  a4paper,
  top=20mm,
  bottom=25mm,
  left=25mm,
  right=25mm
}

% 図を1ページに1つだけ表示するためにplaceinsパッケージを使用
\usepackage{placeins}

% ページスタイルを設定(final = カメラレディ)
\pagestyle{final}

% \maketitleを無効化(NAISTの\titlepageを使用するため)
\makeatletter
\renewcommand{\maketitle}{}
\makeatother

% YAML変数をNAISTのLaTeXコマンドに変換
% プレースホルダーを使用(post-renderスクリプトで展開される)
\makeatletter

% \emptyを定義(空文字列チェック用)
\def\empty{}

% 学籍番号
\studentnumber{123456}

% 修士論文/課題研究(デフォルト: 修士論文 = 1)
\doctitle{\mastersthesis}

% 専攻(デフォルト: 工学 = 1)
\major{\engineering}

% プログラム(デフォルト: 情報理工学プログラム = 1)
\program{\ise}

% タイトル(日本語・英語)
\title{太陽と月を利用した\piの低速計算アルゴリズムに関する理論的研究}
\etitle{Theoretical Studies on Low-Speed Calculation Algorithms of \pi Utilizing the Sun and the Moon}

% 著者(日本語・英語)
\author{田中 研太郎}
\eauthor{Hanako Sentan}

% 提出年月日
\jsyear{令和XX}
\esyear{202X}
\smonth{2}
\sday{20}

% \edatestrを正しく動作させるため、再定義
% naist-mcommon.styの\edatestr定義は\ifcase\smonthstrを使っているが、
% \smonthstrは文字列として定義されているため、数値として扱えない
% そのため、\edatestrを再定義して、月名を直接計算する
% 注意: naist-mcommon.styの読み込み後に再定義する必要があるため、
% ここではプレースホルダーのみ設定し、expand_preamble.pyで実際の値を設定する
\makeatletter
% \edatestr-placeholderを後でexpand_preamble.pyで置換する
\def\edatestr{February 20, 202X}
\makeatother

% naist-mcommon.styの読み込み後に\edatestrを再定義するためのプレースホルダー
% expand_preamble.pyで、naist-mcommon.styの読み込み後に\def\edatestr{...}を追加する

% 研究室名
\jlabname{脳・行動モデリング研究室}
\elabname{xxx Lab.}

% 審査委員(6人まで対応)
% デフォルトで6人表示(空の値の場合は空文字列を使用)
\def\tempthird{あああああ 准教授}
\def\tempfourth{}
\def\tempfifth{}
\def\tempsixth{}
\def\tempthirdpos{(副指導教員,情報科学領域)}
\def\tempfourthpos{}
\def\tempfifthpos{}
\def\tempsixthpos{}
% 条件分岐を削除して、常に6人表示(expand_preamble.pyで値が設定される)
\cmembers{田中 沙織 教授}{(主指導教員,情報科学領域)}
         {あああああ 教授}{(副指導教員,情報科学領域)}
         {あああああ 准教授}{(副指導教員,情報科学領域)}
         {}{}
% 5人目と6人目を追加(\addcmembersコマンドを使用)
\addcmembers{}{}
            {}{}
            {}{}
            {}{}

% 審査委員(英語版、タイトルページで使用)
% english-supervisorが指定されていない場合は、supervisorを使用(暫定的)
\def\tempesupervisor{}
\def\tempecosupervisor{}
\def\tempethird{}
\def\tempethirdpos{}
\def\tempfourth{}
\def\tempfourthpos{}
\ecmembers{\ifx\tempesupervisor\empty Professor 田中 沙織 教授 \else\tempesupervisor\fi}{(Supervisor, Division of Information Science)}
         {\ifx\tempecosupervisor\empty Professor あああああ 教授 \else\tempecosupervisor\fi}{(Co-supervisor, Division of Information Science)}
         {\ifx\tempethird\empty\else\tempethird\fi}{\ifx\tempethirdpos\empty(Co-supervisor, Division of Information Science)\else\tempethirdpos\fi}
         {\ifx\tempfourth\empty\else\tempfourth\fi}{\ifx\tempfourthpos\empty\else\tempfourthpos\fi}

% キーワード
\keywords{$\pi$, 天文学, 数学, 計算機, アルゴリズム}
\ekeywords{$\pi$, astronomy, mathematics, computer, algorithm}

% 概要(abstract)
% \abstractコマンドは既に定義されているため、\renewcommandを使用
\let\oldabstract\abstract
\renewcommand{\abstract}[1]{\def\abstracttext{#1}}
\abstract{人類がこの地上に現われて以来、$\pi$の計算には多くの関心が払われてきた。本論文では、太陽と月を利用して$\pi$を低速に計算するための 画期的なアルゴリズムを与える。ここには内容梗概を書く。ここには内容梗概を書く。ここには内容梗概を書く。 ここには内容梗概を書く。ここには内容梗概を書く。ここには内容梗概を書く。 ここには内容梗概を書く。ここには内容梗概を書く。ここには内容梗概を書く。 ここには内容梗概を書く。ここには内容梗概を書く。ここには内容梗概を書く。 ここには内容梗概を書く。ここには内容梗概を書く。ここには内容梗概を書く。ここには内容梗概を書く。ここには内容梗概を書く。ここには内容梗概を書く。 ここには内容梗概を書く。ここには内容梗概を書く。ここには内容梗概を書く。 ここには内容梗概を書く。ここには内容梗概を書く。ここには内容梗概を書く。 ここには内容梗概を書く。ここには内容梗概を書く。ここには内容梗概を書く。 ここには内容梗概を書く。ここには内容梗概を書く。ここには内容梗概を書く。}

% 英語概要
\eabstract{The calculation of $\pi$ has been paid much attention since human beings appeared on the earth.This thesis presents novel low-speed algorithms to calculate $\pi$ utilizing the sun and the moon.This is a sample abstract. This is a sample abstract. This is a sample abstract. This is a sample abstract. This is a sample abstract. This is a sample abstract. This is a sample abstract. This is a sample abstract. This is a sample abstract. This is a sample abstract.This is a sample abstract. This is a sample abstract. This is a sample abstract. This is a sample abstract. This is a sample abstract. This is a sample abstract. This is a sample abstract. This is a sample abstract. This is a sample abstract. This is a sample abstract.}

\makeatother

% ハイパーリンクの色を黒に設定(目次の文字が青くならないように)
\AtEndPreamble{%
  \hypersetup{
    colorlinks=true,
    linkcolor=black,
    filecolor=black,
    urlcolor=black,
    citecolor=black
  }%
}

% 図のデフォルト配置を独立ページ([p])に設定
% テーブルのデフォルト配置をhtbpに設定(Mtexと同じ)
\makeatletter
\def\fps@figure{p}
\def\fps@table{htbp}
\makeatother

% 図表キャプションの形式をMtexテンプレートに合わせる
% Mtexでは「Figure 1 キャプション」のようにスペースのみ(コロンなし)
% Quartoのデフォルト「Figure 1: キャプション」を上書き
% \figurenameと\tablenameにスペースを追加し、labelsep=spaceでコロンを削除
\makeatletter
\AtBeginDocument{%
  % \figurenameと\tablenameにスペースを追加

  
  \@ifpackageloaded{caption}{%
    % captionパッケージが読み込まれている場合
    % labelsep=spaceでスペースのみ(コロンなし)に設定
    \DeclareCaptionLabelFormat{naist}{#1 #2}
    \captionsetup{labelformat=naist, labelsep=newline, textfont=it, singlelinecheck=false, position=top}
  }{%
    % captionパッケージが読み込まれていない場合(\@makecaptionを使用)
    \long\def\@makecaption#1#2{
     \vskip 10pt 
     \setbox\@tempboxa\hbox{#1 \ #2}
     \ifdim \wd\@tempboxa >\hsize #1 \ #2\par \else \hbox
to\hsize{\hfil\box\@tempboxa\hfil} 
     \fi}
  }%
}
\makeatother

\usepackage{booktabs}
\usepackage{longtable}
\usepackage{array}
\usepackage{multirow}
\usepackage{wrapfig}
\usepackage{float}
\usepackage{colortbl}
\usepackage{pdflscape}
\usepackage{tabu}
\usepackage{threeparttable}
\usepackage{threeparttablex}
\usepackage[normalem]{ulem}
\usepackage{makecell}
\usepackage{xcolor}
\makeatletter
\@ifpackageloaded{caption}{}{\usepackage{caption}}
\AtBeginDocument{%
\ifdefined\contentsname
  \renewcommand*\contentsname{Table of contents}
\else
  \newcommand\contentsname{Table of contents}
\fi
\ifdefined\listfigurename
  \renewcommand*\listfigurename{List of Figures}
\else
  \newcommand\listfigurename{List of Figures}
\fi
\ifdefined\listtablename
  \renewcommand*\listtablename{List of Tables}
\else
  \newcommand\listtablename{List of Tables}
\fi
\ifdefined\figurename
  \renewcommand*\figurename{Figure}
\else
  \newcommand\figurename{Figure}
\fi
\ifdefined\tablename
  \renewcommand*\tablename{Table}
\else
  \newcommand\tablename{Table}
\fi
}
\@ifpackageloaded{float}{}{\usepackage{float}}
\floatstyle{ruled}
\@ifundefined{c@chapter}{\newfloat{codelisting}{h}{lop}}{\newfloat{codelisting}{h}{lop}[chapter]}
\floatname{codelisting}{Listing}
\newcommand*\listoflistings{\listof{codelisting}{List of Listings}}
\makeatother
\makeatletter
\makeatother
\makeatletter
\@ifpackageloaded{caption}{}{\usepackage{caption}}
\@ifpackageloaded{subcaption}{}{\usepackage{subcaption}}
\makeatother

\ifLuaTeX
\usepackage[bidi=basic]{babel}
\else
\usepackage[bidi=default]{babel}
\fi
\babelprovide[main,import]{english}
% get rid of language-specific shorthands (see #6817):
\let\LanguageShortHands\languageshorthands
\def\languageshorthands#1{}
\ifLuaTeX
  \usepackage[english]{selnolig} % disable illegal ligatures
\fi
\usepackage[style=template/jpa,backend=biber]{biblatex}
\addbibresource{bibliography-jp.bib}
\addbibresource{bibliography-en.bib}
\usepackage{bookmark}

\IfFileExists{xurl.sty}{\usepackage{xurl}}{} % add URL line breaks if available
\urlstyle{same} % disable monospaced font for URLs
\hypersetup{
  pdftitle={太陽と月を利用した\textbackslash piの低速計算アルゴリズムに関する理論的研究},
  pdfauthor={田中 研太郎},
  pdflang={en},
  colorlinks=true,
  linkcolor={blue},
  filecolor={Maroon},
  citecolor={Blue},
  urlcolor={Blue},
  pdfcreator={LaTeX via pandoc}}


\title{太陽と月を利用した\(\pi\)の低速計算アルゴリズムに関する理論的研究}
\author{田中 研太郎}
\date{2025-11-27}

\begin{document}
\maketitle

% タイトルページ、審査委員ページ、概要ページを生成(NAISTフォーマット)
\titlepage
\cmemberspage
\firstabstract
\secondabstract

% 目次(NAISTフォーマット)
\toc
% \tocコマンドがtocdepthを2に設定する可能性があるため、再設定
\setcounter{tocdepth}{3}

% 図目次と表目次
\newpage
\listoffigures
\listoftables

% 本文開始
\newpage
\pagenumbering{arabic}

% secnumdepthを設定(Quartoのnumber-depth変数を使用)
% \titlepage、\cmemberspage、\tocなどのコマンドの後に設定することで、
% これらのコマンドがsecnumdepthを上書きした場合でも確実に3に設定される
% 本文が始まる直前に設定することで、subsubsectionの番号が確実に表示される
% _extension.ymlでnumber-depth: 3が設定されているので、デフォルトは3
\setcounter{secnumdepth}{3}


\section{はじめに}\label{ux306fux3058ux3081ux306b}

まず,\textcite{Abrams2020}
のように,すると,bibファイル内のAbramsの2020年の論文が引用されます。そして,次のように,{[}{]}でくくると文末の引用スタイルになります\autocite{Allport1935}。また,文末に複数引用する場合は,こういう感じにします\autocite{Bergson2002,Freud1956}。このQmdファイルではBibLatex-jpaを使っていますので,日本語文献も処理できます。例えば,\textcite{向田2009}
, \textcite{堀2009}, \textcite{矢嶋2013}
は,XXXについて示した\autocite{Helmholtz1925,Freud1956}などの文章も処理できます。

\ref{sec-kako} 節では、過去における研究について述べ、 \ref{sec-kadai}
章では、現状と今後の課題について述べる。 また、付録 \ref{sec-omake1}
におまけその1を添付する。

\subsection{過去における研究}\label{sec-kako}

\begin{figure}[p]

\caption{\label{fig-CNN}Convolutional Neural Network (CNN)}

\centering{

\includegraphics[width=0.8\linewidth,height=\textheight,keepaspectratio]{figures/fig1.png}

}

\end{figure}%

過去における研究としては\autocite{alex_nips12}などがある。

過去における研究 過去における研究 過去における研究 過去における研究
過去における研究 過去における研究 過去における研究 過去における研究
過去における研究 過去における研究 過去における研究

\subsubsection{先行研究での知見1}\label{ux5148ux884cux7814ux7a76ux3067ux306eux77e5ux898b1}

\subsubsection{先行研究での知見2}\label{ux5148ux884cux7814ux7a76ux3067ux306eux77e5ux898b2}

\subsubsection{先行研究での知見3}\label{ux5148ux884cux7814ux7a76ux3067ux306eux77e5ux898b3}

\subsubsection{先行研究での知見4}\label{ux5148ux884cux7814ux7a76ux3067ux306eux77e5ux898b4}

\subsubsection{先行研究での知見5}\label{ux5148ux884cux7814ux7a76ux3067ux306eux77e5ux898b5}

\subsubsection{先行研究での知見6}\label{ux5148ux884cux7814ux7a76ux3067ux306eux77e5ux898b6}

\subsubsection{先行研究での知見7}\label{ux5148ux884cux7814ux7a76ux3067ux306eux77e5ux898b7}

\subsection{先行研究の問題点}\label{sec-mondai}

\subsection{研究の目的と意義}\label{ux7814ux7a76ux306eux76eeux7684ux3068ux610fux7fa9}

研究の目的と意義 研究の目的と意義 研究の目的と意義 研究の目的と意義
研究の目的と意義 研究の目的と意義 研究の目的と意義 研究の目的と意義
研究の目的と意義 研究の目的と意義 研究の目的と意義 研究の目的と意義
研究の目的と意義 研究の目的と意義 研究の目的と意義 研究の目的と意義

\section{現状と今後の課題}\label{sec-kadai}

現状と今後の課題 現状と今後の課題 現状と今後の課題 現状と今後の課題
現状と今後の課題 現状と今後の課題 現状と今後の課題 現状と今後の課題
現状と今後の課題 現状と今後の課題 現状と今後の課題 現状と今後の課題
現状と今後の課題 現状と今後の課題 現状と今後の課題 現状と今後の課題

\section{方法}\label{ux65b9ux6cd5}

\subsection{参加者}\label{ux53c2ux52a0ux8005}

\subsection{実験材料・装置・刺激}\label{ux5b9fux9a13ux6750ux6599ux88c5ux7f6eux523aux6fc0}

\begin{figure}[p]

\caption{\label{fig-muller-lyer}ミュラー・リヤー錯視の例}

\centering{

\includegraphics[width=0.8\linewidth,height=\textheight,keepaspectratio]{figures/fig1.png}

}

\end{figure}%

\subsection{統計解析}\label{ux7d71ux8a08ux89e3ux6790}

 統計解析は,macOS 15.3.1上で,R version 4.3.3
(2024-02-29)を用いて実施された。

\section{結果}\label{ux7d50ux679c}

\subsection{記述統計}\label{ux8a18ux8ff0ux7d71ux8a08}

\begin{table}

\caption{Big Five因子の記述統計量}
\centering
\begin{tabular}[t]{lrrrrrrrr}
\toprule
  & n & Mean & SD & Median & Min & Max & Skewness & kurtosis\\
\midrule
Extraversion & 2713 & 18.96 & 2.71 & 19 & 5 & 29 & 0.01 & 1.08\\
Neuroticism & 2694 & 15.82 & 5.97 & 15 & 5 & 30 & 0.22 & -0.66\\
Conscientiousness & 2707 & 19.04 & 2.77 & 19 & 5 & 30 & -0.17 & 0.81\\
Agreeableness & 2709 & 21.04 & 3.68 & 22 & 5 & 30 & -0.66 & 0.68\\
Openness & 2726 & 19.34 & 2.74 & 19 & 5 & 29 & -0.02 & 1.09\\
\bottomrule
\multicolumn{9}{l}{\textsuperscript{} Note. SD=standard deviation}\\
\end{tabular}
\end{table}

\subsection{メインの解析の前提となる解析}\label{ux30e1ux30a4ux30f3ux306eux89e3ux6790ux306eux524dux63d0ux3068ux306aux308bux89e3ux6790}

\subsubsection{変数間の相関係数}\label{ux5909ux6570ux9593ux306eux76f8ux95a2ux4fc2ux6570}

\renewcommand{\arraystretch}{1}
\begin{landscape}\begin{table}

\caption{Big Five因子の平均・標準偏差と相関}
\fontsize{10}{12}\selectfont
\begin{threeparttable}
\begin{tabular}[t]{lrrccccc}
\toprule
Variable & $N$ & $M$ & $SD$ & 1 & 2 & 3 & 4\\
\midrule
1. Extraversion & 2713 & 18.96 & 2.71 &  &  &  & \\
 &  &  &  &  &  &  & \\
 &  &  &  &  &  &  & \\
 &  &  &  &  &  &  & \\
2. Neuroticism & 2694 & 15.82 & 5.97 & .04$^{*}$ &  &  & \\
 &  &  &  & {}[.00, .08] &  &  & \\
 &  &  &  & $p$ = .049 &  &  & \\
 &  &  &  &  &  &  & \\
3. Conscientiousness & 2707 & 19.04 & 2.77 & .18$^{**}$ & .25$^{**}$ &  & \\
 &  &  &  & {}[.15, .22] & {}[.21, .28] &  & \\
 &  &  &  & $p$ < .001 & $p$ < .001 &  & \\
 &  &  &  &  &  &  & \\
4. Agreeableness & 2709 & 21.04 & 3.68 & .30$^{**}$ & -.14$^{**}$ & .06$^{**}$ & \\
 &  &  &  & {}[.26, .33] & {}[-.18, -.10] & {}[.02, .10] & \\
 &  &  &  & $p$ < .001 & $p$ < .001 & $p$ = .002 & \\
 &  &  &  &  &  &  & \\
5. Openness & 2726 & 19.34 & 2.74 & .25$^{**}$ & .16$^{**}$ & .25$^{**}$ & .19$^{**}$\\
 &  &  &  & {}[.22, .29] & {}[.12, .20] & {}[.21, .28] & {}[.15, .23]\\
 &  &  &  & $p$ < .001 & $p$ < .001 & $p$ < .001 & $p$ < .001\\
 &  &  &  &  &  &  & \\
\bottomrule
\end{tabular}
\begin{tablenotes}
\item \textit{Note}. \textit{N} = number of cases. \textit{M} = mean. \textit{SD} = standard deviation. Square brackets = 95\% confidence interval. \newline  * indicates $p$ < .05. ** indicates $p$ < .01.
\end{tablenotes}
\end{threeparttable}
\end{table}
\end{landscape}
\renewcommand{\arraystretch}{1}

\subsubsection{ヒストグラム}\label{ux30d2ux30b9ux30c8ux30b0ux30e9ux30e0}

\begin{figure}[H]

\caption{\label{fig:figs}神経症傾向のヒストグラム}

{\centering \pandocbounded{\includegraphics[keepaspectratio]{paper_files/figure-pdf/unnamed-chunk-4-1.pdf}}

}

\end{figure}%

\subsubsection{2群の比較(連続変数)}\label{ux7fa4ux306eux6bd4ux8f03ux9023ux7d9aux5909ux6570}

神経症傾向に関して性差を検討したところ,男性(\emph{M} = 14.74, \emph{SD}
= 5.72)よりも、女性(\emph{M} = 16.35, \emph{SD} =
6.03)の方が有意に神経症傾向が高かった(\emph{t} (1853.20) = 6.77,
\emph{p} = 0.00 , \emph{d} = 0.27,95 \%CI {[}0.19, 0.35{]})。

\begin{figure}[H]

\caption{\label{fig:figs}神経症傾向の平均と}

{\centering \pandocbounded{\includegraphics[keepaspectratio]{paper_files/figure-pdf/figs-1.pdf}}

}

\end{figure}%

\subsubsection{2群の比較(離散変数)}\label{ux7fa4ux306eux6bd4ux8f03ux96e2ux6563ux5909ux6570}

\begin{table}

\caption{性別と教育歴についてのクロス集計表}
\centering
\begin{tabular}[t]{lrr}
\toprule
\multicolumn{1}{c}{ } & \multicolumn{2}{c}{Education} \\
\cmidrule(l{3pt}r{3pt}){2-3}
  & High & Low\\
\midrule
\addlinespace[0.3em]
\multicolumn{3}{l}{\textbf{Gender}}\\
\hspace{1em}Female & 131 & 1608\\
\hspace{1em}Male & 93 & 745\\
\bottomrule
\end{tabular}
\end{table}

女性より、男性の方が高学歴者が多いことが示唆された( \(\chi ^2\) (1.00,
\emph{N} = 2800) = 8.61, \emph{p} = 0.00)。

\subsection{メインの解析の記載}\label{ux30e1ux30a4ux30f3ux306eux89e3ux6790ux306eux8a18ux8f09}

\subsubsection{重回帰分析}\label{ux91cdux56deux5e30ux5206ux6790}

\renewcommand{\arraystretch}{1}
\begin{landscape}\begin{table}

\caption{重回帰分析結果}
\fontsize{10}{12}\selectfont
\begin{threeparttable}
\begin{tabular}[t]{lrrcrcrcc}
\toprule
Predictor & $b$ & 95\% CI & $beta$ & 95\% CI & Unique $R^2$ & 95\% CI & $r$ & Fit\\
\midrule
(Intercept) & 41.12** & {}[22.72, 59.53] &  &  &  &  &  & \\
adverts & 0.09** & {}[0.07, 0.10] & 0.52 & {}[0.44, 0.61] & .27** & {}[.18, .36] & .58** & \\
airplay & 3.59** & {}[3.02, 4.15] & 0.55 & {}[0.46, 0.63] & .29** & {}[.20, .38] & .60** & \\
 &  &  &  &  &  &  &  & $R^2$ = .629**\\
 &  &  &  &  &  &  &  & 95\% CI[.55,.69]\\
 &  &  &  &  &  &  &  & \\
\bottomrule
\end{tabular}
\begin{tablenotes}
\item \textit{Note}. $N$ = 200. $b$ = unstandardized regression weight. $beta$ = standardized regression weight. Unique $R^2$ = semipartial correlation squared. $r$ = zero-order correlation. CI = confidence interval. \newline  * indicates $p$ < .05. ** indicates $p$ < .01.
\end{tablenotes}
\end{threeparttable}
\end{table}
\end{landscape}
\renewcommand{\arraystretch}{1}

\subsubsection{分散分析の結果}\label{ux5206ux6563ux5206ux6790ux306eux7d50ux679c}

\renewcommand{\arraystretch}{1}\begin{table}

\caption{分散分析結果}
\fontsize{10}{12}\selectfont
\begin{threeparttable}
\begin{tabular}[t]{lrrrrrrc}
\toprule
Predictor & $SS$ & $df$ & $MS$ & $F$ & $p$ & $\eta_{partial}^2$ & 90\% CI\\
\midrule
(Intercept) & 29403.12 & 1 & 29403.12 & 354.10 & <.001 &  & \\
gender & 156.25 & 1 & 156.25 & 1.88 & .177 & .04 & {}[.00, .17]\\
alcohol & 102.08 & 2 & 51.04 & 0.61 & .546 & .03 & {}[.00, .12]\\
gender x alcohol & 1978.12 & 2 & 989.06 & 11.91 & <.001 & .36 & {}[.15, .49]\\
Error & 3487.50 & 42 & 83.04 &  &  &  & \\
\bottomrule
\end{tabular}
\begin{tablenotes}
\item \textit{Note}. $SS$ = Sum of squares. $df$ = degrees of freedom. $MS$ = mean square. CI indicates the confidence interval for $\eta_{partial}^2$.
\end{tablenotes}
\end{threeparttable}
\end{table}
\renewcommand{\arraystretch}{1}

\subsection{メインの解析結果を補強する解析の記載}\label{ux30e1ux30a4ux30f3ux306eux89e3ux6790ux7d50ux679cux3092ux88dcux5f37ux3059ux308bux89e3ux6790ux306eux8a18ux8f09}

\section{考察}\label{ux8003ux5bdf}

\subsection{主要な発見の概要}\label{ux4e3bux8981ux306aux767aux898bux306eux6982ux8981}

\subsection{考えられるメカニズムの考察と説明}\label{ux8003ux3048ux3089ux308cux308bux30e1ux30abux30cbux30baux30e0ux306eux8003ux5bdfux3068ux8aacux660e}

\subsection{関連のある先行研究の結果との比較}\label{ux95a2ux9023ux306eux3042ux308bux5148ux884cux7814ux7a76ux306eux7d50ux679cux3068ux306eux6bd4ux8f03}

\subsection{研究結果が与える示唆}\label{ux7814ux7a76ux7d50ux679cux304cux4e0eux3048ux308bux793aux5506}

\subsection{研究の限界と今後の課題}\label{ux7814ux7a76ux306eux9650ux754cux3068ux4ecaux5f8cux306eux8ab2ux984c}

\subsection{結論}\label{ux7d50ux8ad6}

\section{要約}\label{ux8981ux7d04}

\section{謝辞}\label{ux8b1dux8f9e}

Thank you. Thank you.

\section{参考文献}\label{ux53c2ux8003ux6587ux732e}

\printbibliography[heading=none]

\appendix

\section{付録}\label{ux4ed8ux9332}

\subsection{おまけその1}\label{sec-omake1}

これはおまけです。これはおまけです。これはおまけです。これはおまけです。
これはおまけです。これはおまけです。これはおまけです。これはおまけです。
これはおまけです。これはおまけです。これはおまけです。これはおまけです。
これはおまけです。これはおまけです。これはおまけです。これはおまけです。

\begin{figure}
\centerline{これはおまけの図です。}
\caption{おまけの図}
\end{figure}

\subsection{おまけその2}\label{ux304aux307eux3051ux305dux306euxff12}

これもおまけです。これもおまけです。これもおまけです。これもおまけです。
これもおまけです。これもおまけです。これもおまけです。これもおまけです。
これもおまけです。これもおまけです。これもおまけです。これもおまけです。
これもおまけです。これもおまけです。これもおまけです。これもおまけです。


% Empty biblio.tex partial to suppress default \printbibliography
% This prevents Quarto from automatically adding a "References" section
% when using BibLaTeX. Manual \printbibliography[heading=none] in the document
% will still work.


\end{document}
