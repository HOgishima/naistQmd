% NAIST修士論文用LaTeXヘッダー(Quarto用)
% senshuQmdスタイルを参考に、NAISTフォーマットに対応

% Quartoのデフォルト\maketitleを先にバックアップ(NAISTの\titlepageを使用するため)
\makeatletter
\@ifundefined{maketitle}{}{%
  \let\quarto@maketitle\maketitle
}
\makeatother

% NAISTのスタイルファイルを読み込む
% template/naist-jmthesis.styを使用(\input naist-mcommon.styを\input{Mtex/naist-mcommon.sty}に修正済み)
\makeatletter
\input{template/naist-jmthesis.sty}
\makeatother

% NAISTフォーマットのページレイアウトを設定(スタイルファイルの後に設定して上書き)
\usepackage{geometry}
\geometry{
  a4paper,
  top=20mm,
  bottom=25mm,
  left=25mm,
  right=25mm
}

% ページスタイルを設定(final = カメラレディ)
\pagestyle{final}

% \maketitleを無効化(NAISTの\titlepageを使用するため)
\makeatletter
\renewcommand{\maketitle}{}
\makeatother

% YAML変数をNAISTのLaTeXコマンドに変換
% プレースホルダーを使用(post-renderスクリプトで展開される)
\makeatletter

% 学籍番号
\studentnumber{$student-id$}

% 修士論文/課題研究(デフォルト: 修士論文 = 1)
\doctitle{\mastersthesis}

% 専攻(デフォルト: 工学 = 1)
\major{\engineering}

% プログラム(デフォルト: 情報理工学プログラム = 1)
\program{\ise}

% タイトル(日本語・英語)
\title{$japanese-title$}
\etitle{$english-title$}

% 著者(日本語・英語)
\author{$japanese-author$}
\eauthor{$english-author$}

% 提出年月日
\jsyear{$japanese-year$}
\esyear{$english-year$}
\smonth{$submission-month$}
\sday{$submission-day$}

% \edatestrを正しく動作させるため、再定義
% naist-mcommon.styの\edatestr定義は\ifcase\smonthstrを使っているが、
% \smonthstrは文字列として定義されているため、数値として扱えない
% そのため、\edatestrを再定義して、月名を直接計算する
\makeatletter
% \edatestr-placeholderを後でexpand_preamble.pyで置換する
\def\edatestr{$edatestr-placeholder$}
\makeatother

% 研究室名
\jlabname{$lab-name-japanese$}
\elabname{$lab-name-english$}

% 審査委員(4人まで対応)
% 空の値の場合は\emptyを使用
\def\tempthird{$third-member$}
\def\tempfourth{$fourth-member$}
\def\tempthirdpos{$third-position$}
\def\tempfourthpos{$fourth-position$}
\cmembers{$supervisor$}{(主指導教員,情報科学領域)}
         {$co-supervisor$}{(副指導教員,情報科学領域)}
         {\ifx\tempthird\empty\else\tempthird\fi}{\ifx\tempthirdpos\empty\else\tempthirdpos\fi}
         {\ifx\tempfourth\empty\else\tempfourth\fi}{\ifx\tempfourthpos\empty\else\tempfourthpos\fi}

% キーワード
\keywords{$keywords-japanese$}
\ekeywords{$keywords-english$}

% 概要(abstract)
% \abstractコマンドは既に定義されているため、\renewcommandを使用
\let\oldabstract\abstract
\renewcommand{\abstract}[1]{\def\abstracttext{#1}}
\abstract{$japanese-abstract$}

% 英語概要
\eabstract{$english-abstract$}

\makeatother

% ハイパーリンクの色を黒に設定(目次の文字が青くならないように)
\AtEndPreamble{%
  \hypersetup{
    colorlinks=true,
    linkcolor=black,
    filecolor=black,
    urlcolor=black,
    citecolor=black
  }%
}

