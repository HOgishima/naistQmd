% NAIST修士論文用LaTeXヘッダー(Quarto用)
% senshuQmdスタイルを参考に、NAISTフォーマットに対応

% Quartoのデフォルト\maketitleを先にバックアップ(NAISTの\titlepageを使用するため)
\makeatletter
\@ifundefined{maketitle}{}{%
  \let\quarto@maketitle\maketitle
}
\makeatother

% NAISTのスタイルファイルを読み込む
% template/naist-jmthesis.styを使用(\input naist-mcommon.styを\input{Mtex/naist-mcommon.sty}に修正済み)
\makeatletter
\input{template/naist-jmthesis.sty}
\makeatother

% NAISTフォーマットのページレイアウトを設定(スタイルファイルの後に設定して上書き)
\usepackage{geometry}
\geometry{
  a4paper,
  top=20mm,
  bottom=25mm,
  left=25mm,
  right=25mm
}

% 図を1ページに1つだけ表示するためにplaceinsパッケージを使用
\usepackage{placeins}

% ページスタイルを設定(final = カメラレディ)
\pagestyle{final}

% \maketitleを無効化(NAISTの\titlepageを使用するため)
\makeatletter
\renewcommand{\maketitle}{}
\makeatother

% YAML変数をNAISTのLaTeXコマンドに変換
% プレースホルダーを使用(post-renderスクリプトで展開される)
\makeatletter

% \emptyを定義(空文字列チェック用)
\def\empty{}

% 学籍番号
\studentnumber{123456}

% 修士論文/課題研究(デフォルト: 修士論文 = 1)
\doctitle{\mastersthesis}

% 専攻(デフォルト: 工学 = 1)
\major{\engineering}

% プログラム(デフォルト: 情報理工学プログラム = 1)
\program{\ise}

% タイトル(日本語・英語)
\title{太陽と月を利用した\piの低速計算アルゴリズムに関する理論的研究}
\etitle{Theoretical Studies on Low-Speed Calculation Algorithms of \pi Utilizing the Sun and the Moon}

% 著者(日本語・英語)
\author{田中 研太郎}
\eauthor{Hanako Sentan}

% 提出年月日
\jsyear{令和XX}
\esyear{202X}
\smonth{2}
\sday{20}

% \edatestrを正しく動作させるため、再定義
% naist-mcommon.styの\edatestr定義は\ifcase\smonthstrを使っているが、
% \smonthstrは文字列として定義されているため、数値として扱えない
% そのため、\edatestrを再定義して、月名を直接計算する
% 注意: naist-mcommon.styの読み込み後に再定義する必要があるため、
% ここではプレースホルダーのみ設定し、expand_preamble.pyで実際の値を設定する
\makeatletter
% \edatestr-placeholderを後でexpand_preamble.pyで置換する
\def\edatestr{February 20, 202X}
\makeatother

% naist-mcommon.styの読み込み後に\edatestrを再定義するためのプレースホルダー
% expand_preamble.pyで、naist-mcommon.styの読み込み後に\def\edatestr{...}を追加する

% 研究室名
\jlabname{脳・行動モデリング研究室}
\elabname{xxx Lab.}

% 審査委員(6人まで対応)
% デフォルトで6人表示(空の値の場合は空文字列を使用)
\def\tempthird{あああああ 准教授}
\def\tempfourth{}
\def\tempfifth{}
\def\tempsixth{}
\def\tempthirdpos{(副指導教員,情報科学領域)}
\def\tempfourthpos{}
\def\tempfifthpos{}
\def\tempsixthpos{}
% 条件分岐を削除して、常に6人表示(expand_preamble.pyで値が設定される)
\cmembers{田中 沙織 教授}{(主指導教員,情報科学領域)}
         {あああああ 教授}{(副指導教員,情報科学領域)}
         {あああああ 准教授}{(副指導教員,情報科学領域)}
         {}{}
% 5人目と6人目を追加(\addcmembersコマンドを使用)
\addcmembers{}{}
            {}{}
            {}{}
            {}{}

% 審査委員(英語版、タイトルページで使用)
% english-supervisorが指定されていない場合は、supervisorを使用(暫定的)
\def\tempesupervisor{}
\def\tempecosupervisor{}
\def\tempethird{}
\def\tempethirdpos{}
\def\tempfourth{}
\def\tempfourthpos{}
\ecmembers{\ifx\tempesupervisor\empty Professor 田中 沙織 教授 \else\tempesupervisor\fi}{(Supervisor, Division of Information Science)}
         {\ifx\tempecosupervisor\empty Professor あああああ 教授 \else\tempecosupervisor\fi}{(Co-supervisor, Division of Information Science)}
         {\ifx\tempethird\empty\else\tempethird\fi}{\ifx\tempethirdpos\empty(Co-supervisor, Division of Information Science)\else\tempethirdpos\fi}
         {\ifx\tempfourth\empty\else\tempfourth\fi}{\ifx\tempfourthpos\empty\else\tempfourthpos\fi}

% キーワード
\keywords{$\pi$, 天文学, 数学, 計算機, アルゴリズム}
\ekeywords{$\pi$, astronomy, mathematics, computer, algorithm}

% 概要(abstract)
% \abstractコマンドは既に定義されているため、\renewcommandを使用
\let\oldabstract\abstract
\renewcommand{\abstract}[1]{\def\abstracttext{#1}}
\abstract{人類がこの地上に現われて以来、$\pi$の計算には多くの関心が払われてきた。本論文では、太陽と月を利用して$\pi$を低速に計算するための 画期的なアルゴリズムを与える。ここには内容梗概を書く。ここには内容梗概を書く。ここには内容梗概を書く。 ここには内容梗概を書く。ここには内容梗概を書く。ここには内容梗概を書く。 ここには内容梗概を書く。ここには内容梗概を書く。ここには内容梗概を書く。 ここには内容梗概を書く。ここには内容梗概を書く。ここには内容梗概を書く。 ここには内容梗概を書く。ここには内容梗概を書く。ここには内容梗概を書く。ここには内容梗概を書く。ここには内容梗概を書く。ここには内容梗概を書く。 ここには内容梗概を書く。ここには内容梗概を書く。ここには内容梗概を書く。 ここには内容梗概を書く。ここには内容梗概を書く。ここには内容梗概を書く。 ここには内容梗概を書く。ここには内容梗概を書く。ここには内容梗概を書く。 ここには内容梗概を書く。ここには内容梗概を書く。ここには内容梗概を書く。}

% 英語概要
\eabstract{The calculation of $\pi$ has been paid much attention since human beings appeared on the earth.This thesis presents novel low-speed algorithms to calculate $\pi$ utilizing the sun and the moon.This is a sample abstract. This is a sample abstract. This is a sample abstract. This is a sample abstract. This is a sample abstract. This is a sample abstract. This is a sample abstract. This is a sample abstract. This is a sample abstract. This is a sample abstract.This is a sample abstract. This is a sample abstract. This is a sample abstract. This is a sample abstract. This is a sample abstract. This is a sample abstract. This is a sample abstract. This is a sample abstract. This is a sample abstract. This is a sample abstract.}

\makeatother

% ハイパーリンクの色を黒に設定(目次の文字が青くならないように)
\AtEndPreamble{%
  \hypersetup{
    colorlinks=true,
    linkcolor=black,
    filecolor=black,
    urlcolor=black,
    citecolor=black
  }%
}

% 図のデフォルト配置を独立ページ([p])に設定
% テーブルのデフォルト配置をhtbpに設定(Mtexと同じ)
\makeatletter
\def\fps@figure{p}
\def\fps@table{htbp}
\makeatother

% 図表キャプションの形式をMtexテンプレートに合わせる
% Mtexでは「Figure 1 キャプション」のようにスペースのみ(コロンなし)
% Quartoのデフォルト「Figure 1: キャプション」を上書き
% \figurenameと\tablenameにスペースを追加し、labelsep=spaceでコロンを削除
\makeatletter
\AtBeginDocument{%
  % \figurenameと\tablenameにスペースを追加

  
  \@ifpackageloaded{caption}{%
    % captionパッケージが読み込まれている場合
    % labelsep=spaceでスペースのみ(コロンなし)に設定
    \DeclareCaptionLabelFormat{naist}{#1 #2}
    \captionsetup{labelformat=naist, labelsep=newline, textfont=it, singlelinecheck=false, position=top}
  }{%
    % captionパッケージが読み込まれていない場合(\@makecaptionを使用)
    \long\def\@makecaption#1#2{
     \vskip 10pt 
     \setbox\@tempboxa\hbox{#1 \ #2}
     \ifdim \wd\@tempboxa >\hsize #1 \ #2\par \else \hbox
to\hsize{\hfil\box\@tempboxa\hfil} 
     \fi}
  }%
}
\makeatother

